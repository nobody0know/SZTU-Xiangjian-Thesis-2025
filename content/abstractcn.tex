%!TEX root = ../sztuthesis_main.tex
% 设置中文摘要

\phantomsection
\addcontentsline{toc}{section}{摘要}

\keywordscn{深圳技术大学;心电采集;无线同步;局域网;多设备;时间同步}
\categorycn{TP391}
\begin{abstractcn}
日益上升的心血管疾病发病率凸显了心电监测技术发展的重要性,传统的心电图机检测在便捷性和实时性方面存在一定局限性,尤其是当用户需要随身持续检测的工况下。
本项目针对多设备心电信号采集中的各关键问题,尤其是设备数量、时间同步及并发测量的问题展开研究,成功实现了局域网内大量心电采集设备的低延迟高速率同步采集功能。

系统由实现心电采集与无线射频发送的节点设备、实现时钟同步与组网并通过LVGL图形框架显示基本数据的基站设备和由QT编写图形化上位机数据收集监控与控制软件组成。
节点设备采用 ESP32-C3 芯片实现心电信号采集与传输,经板载硬件模拟前端滤波、放大等处理后使得心电信号幅值处于1.5-2.5V之间,通过ESP32-C3的ADC部分采集放大后的心电信号电压幅值,以约 1000Hz 速率采集并发送数据;
基站设备提供时钟信号与网络管理;上位机程序可灵活配置设备、展示和存储数据。

测试表明,数据实际采集速率达 990Hz,心电图波形清晰,节点设备平均工作电流约 100mA,采用两节10400电池即可持续工作6小时。

本研究为无线心电同步采集技术发展提供了有效技术方案,对于心血管疾病诊断和长期健康监测具有一定意义。

% 图X幅,表X个,参考文献X篇(四号宋体)

\end{abstractcn}