%!TEX root = ../sztuthesis_main.tex
% 设置中文摘要

\phantomsection
\addcontentsline{toc}{section}{摘要}

\keywordscn{深圳技术大学;心电采集;无线同步;局域网;多设备;时间同步}
\categorycn{TP391}
\begin{abstractcn}
心血管疾病发病率的上升凸显了心电监测的重要性,传统心电图检测技术在便捷性、准确性和实时性方面存在局限。
本项目针对多设备心电信号采集的时间同步及并发测量问题展开研究,成功实现了局域网内大量心电采集设备的低延迟高速率同步采集功能。
系统由心电采集与发送节点设备、时钟同步与组网基站设备和上位机数据收集监控与控制软件组成。
节点设备采用 ESP32C3 芯片实现心电信号采集与传输,经滤波、放大等处理后以约 1000Hz 速率采集并发送数据;
基站设备提供时钟信号与网络管理;上位机程序可灵活配置设备、展示和存储数据。
测试表明,数据采集速率达 1000Hz,心电图波形清晰,节点设备平均工作电流约 100mA,采用两节10400电池可持续工作6小时。
本研究为无线心电同步采集技术发展提供了有效方案,对心血管疾病诊断具有重要意义。

% 图X幅,表X个,参考文献X篇(四号宋体)

\end{abstractcn}