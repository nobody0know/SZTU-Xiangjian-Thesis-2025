%!TEX root = ../sztuthesis_main.tex
% 设置中文摘要

\phantomsection
\addcontentsline{toc}{section}{摘要}

\keywordscn{心电采集;无线同步;局域网;多设备;时间同步}
\categorycn{TP391}
\begin{abstractcn}
日益上升的心血管疾病发病率凸显了心电监测技术发展的重要性,传统的心电图机检测在便捷性和实时性方面存在一定局限性,尤其是当用户需要随身持续检测的工况下。
本项目针对多设备心电信号采集中的各关键问题,尤其是设备数量、时间同步及并发测量的问题展开研究,成功实现了局域网内大量心电采集设备的低延迟高速率同步采集功能。

系统由实现心电采集与无线射频发送的节点设备、实现时钟同步与组网并通过LVGL图形框架显示基本数据的基站设备和由QT编写图形化上位机数据收集监控与控制软件组成。
节点设备采用 ESP32-C3 芯片实现心电信号采集与传输,经板载硬件模拟前端滤波、放大等处理后使得心电信号幅值处于1.5-2.5V之间,通过ESP32-C3的ADC部分采集放大后的心电信号电压幅值,以约 1000Hz 速率采集并发送数据;
基站设备提供时钟信号与网络管理;上位机程序可灵活配置设备、展示和存储数据。

本研究通过软硬件协同优化设计,成功构建了支持多节点协同采集的无线心电监测系统。系统创新性地采用基于ESP-NOW协议配网+UDP传输的双层网络架构,在局域网环境下实现了多台设备的毫秒级时间同步(同步误差<±2ms),突破了传统单设备采集的时空限制。测试数据表明,系统在15节点并发工况下仍能维持990Hz的有效采样率,UDP数据包丢失率低于0.01\%,时间戳连续率超过99\%。通过引入自适应滤波算法与右腿驱动技术,在复杂电磁环境中仍可保持-19.66dB的信噪比,显著优于同类无线心电设备(典型值-15dB)。功耗管理方面,创新性地融合Auto Light-Sleep模式与动态电压调频技术,使节点设备在100mA平均工作电流下实现6小时持续监测,较传统方案续航提升40\%。

该技术方案在运动负荷试验和睡眠监测场景中展现出独特优势。系统支持CSV数据导出,为远程心电监护和AI辅助诊断提供了高质量数据源。研究成果不仅推动了可穿戴医疗设备的技术革新,更为构建区域性心电监测网络、实现心血管疾病群防群控提供了重要技术支撑,对分级诊疗体系建设和医疗资源下沉具有积极意义。

% 图X幅,表X个,参考文献X篇(四号宋体)

\end{abstractcn}