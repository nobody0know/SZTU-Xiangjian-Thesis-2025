%!TEX root = ../sztuthesis_main.tex
% 论文正文是主体,主体部分应从另页右页开始,每一章应另起页。一般由序号标题、文字叙述、图、表格和公式等五个部分构成。
\section{引言}
\subsection{研究背景与意义}

心血管疾病(CVD),包括心脏病、高血压、心律失常等症状,已成为全球死亡的主要原因之一。据《中国心血管健康与疾病报告2022概要》的数据,我国CVD的发病率以及死亡率不断提高,报告推算,我国CVD患者现患人数为3.3亿 \cite{中国心血管健康与疾病报告2022概要} ,我国城乡居民人口因心血管疾病造成的死亡约占城乡居民疾病死亡构成比的二分之一。在此背景下,早期诊断和及时治疗心脏病变,尤其是通过有效的心电监测手段,已成为拯救患者生命的关键。

\begin{figure}[hbt]
    \centering
    \includegraphics[width=0.5\textwidth]{latex.jpg}
    \caption{LaTeX}
    \label{F.latex}
\end{figure}

插入公式,对 \app{Word} 小白来说,公式居中编号靠右就是一道百度搜索能力过滤器。

\app{Word} 里编辑三线表,啊烦躁。

等等等等……

让我们,专心写论文好不好?

爱你们。


\subsection{主要研究工作}
虽然我 \LaTeX 水平也很水……但是通过大量 \oper{debug} 也勉强给大家凑出来一个格式绝对标准的 \LaTeX 模板,模板代码丑就丑吧,能用就行。写了大量注释,有一点 \LaTeX 基础就可以根据自己需要修改 \cls{SZTUthesis.cls} 文件。

(1) 提供图片插入示例。

(2) 提供表格插入示例。

(3) 提供公式插入示例。

(4) 提供参考文献插入示例。

\subsection{论文组织结构}

全文内容共六章,具体内容组织如下:

第一章为绪论。

第二章为图片插入示例。

第三章为表格插入示例。

第四章为公式插入示例。

第五章为参考文献插入示例。

第六章总结与展望,总结了本文的主要工作,展望了下一阶段的研究方向。

% \newpage    % 两个章节之间分页,不想分的话可注释掉

\section{图像布局}
\label{sec.figure}

\emph{学校对图片只有小标题要求,没有进一步的子图要求,我们按科技论文常规排版来}

\subsection{单图布局}

\lipsum

\emph{单图布局如图~\ref{F.sztu_single} 所示。}

\begin{figure}[hbt]
\centering
\includegraphics[width=0.5\textwidth]{sztu.png}
\caption{单图布局示例}
\label{F.sztu_single}
\end{figure}

\subsection{横排布局}

\emph{横排布局如图~\ref{F.sztu_row} 所示。}

\begin{figure}[!htb]
    \centering
    \begin{subfigure}[t]{0.24\linewidth}
        \begin{minipage}[b]{1\linewidth}
        \includegraphics[width=1\linewidth]{sztu.png}
        \caption{可以增加描述}
        \end{minipage}
    \end{subfigure}
    \begin{subfigure}[t]{0.24\linewidth}
        \begin{minipage}[b]{1\linewidth}
        \includegraphics[width=1\linewidth]{sztu.png}
        \caption{}
        \end{minipage}
    \end{subfigure}
    \begin{subfigure}[t]{0.24\linewidth}
        \begin{minipage}[b]{1\linewidth}
        \includegraphics[width=1\linewidth]{sztu.png}
        \caption{}
        \end{minipage}
    \end{subfigure}
    \begin{subfigure}[t]{0.24\linewidth}
        \begin{minipage}[b]{1\linewidth}
        \includegraphics[width=1\linewidth]{sztu.png}
        \caption{}
        \end{minipage}
    \end{subfigure}
    \caption{横排布局示例}
    \label{F.sztu_row}
\end{figure}

\lipsum

\subsection{竖排布局}
\emph{竖排布局如图\ref{F.sztu_col}所示。}

\begin{figure}[!htb]
    \centering
    \begin{subfigure}[t]{0.15\linewidth}
        \begin{minipage}[b]{1\linewidth}
        \includegraphics[width=1\linewidth]{sztu.png}
        \caption{}
        \end{minipage}
    \end{subfigure}\\
    \begin{subfigure}[t]{0.15\linewidth}
        \begin{minipage}[b]{1\linewidth}
        \includegraphics[width=1\linewidth]{sztu.png}
        \caption{}
        \end{minipage}
    \end{subfigure}
    \caption{竖排布局示例}
    \label{F.sztu_col}
\end{figure}

\lipsum

\subsection{竖排多图横排布局}

\begin{figure}[!htb]
    \centering
    \begin{subfigure}[t]{0.13\linewidth}
        \begin{minipage}[b]{1\linewidth}
        \includegraphics[width=1\linewidth]{sztu.png} \vspace{-1ex} \vfill
        \includegraphics[width=1\linewidth]{sztu.png}
        \end{minipage}
        \caption{}
    \end{subfigure}
    \begin{subfigure}[t]{0.13\linewidth}
        \begin{minipage}[b]{1\linewidth}
        \includegraphics[width=1\linewidth]{sztu.png} \vspace{-1ex} \vfill
        \includegraphics[width=1\linewidth]{sztu.png}
        \end{minipage}
        \caption{}
    \end{subfigure}
    \caption{竖排多图横排布局}
    \label{F.sztu_col_row}
\end{figure}

\emph{竖排多图横排布局如图~\ref{F.sztu_col_row} 所示。注意看(a)、(b)编号与图关系。}


\subsection{横排多图竖排布局}

\lipsum

\begin{figure}[!htb]
    \centering
    \begin{subfigure}[t]{0.3\linewidth}
        \begin{minipage}[b]{1\linewidth}
        \includegraphics[width=0.45\linewidth]{sztu.png}
        \includegraphics[width=0.45\linewidth]{sztu.png}
        \end{minipage}
        \caption{}
    \end{subfigure}\\
    \begin{subfigure}[t]{0.3\linewidth}
        \begin{minipage}[b]{1\linewidth}
        \includegraphics[width=0.45\linewidth]{sztu.png}
        \includegraphics[width=0.45\linewidth]{sztu.png}
        \end{minipage}
        \caption{}
    \end{subfigure}
    \caption{横排多图竖排布局}
    \label{F.sztu_row_col}
\end{figure}

\emph{横排多图竖排布局如图~\ref{F.sztu_row_col} 所示。注意看(a)、(b)编号与图关系。}

\subsection{本章小结}
本章示例图片布局。

这里再测试一下不同章节的公式编号
\begin{equation}
p_{i} = \frac{e^{-\varepsilon_{i}/kT}}{\sum_{j=1}^{M}e^{-\varepsilon_{j}/kT}}
\end{equation}

\newpage    % 两个章节之间分页,不想分的话可注释掉


\section{表格插入示例}

\begin{table}[htb]
  \centering
  \caption{学校文件里对表格的要求不是很高,不过按照学术论文的一般规范,表格为三线表。}
  \label{T.example}
  \begin{tabular}{llllll}
  \hline
   & A  & B  & C  & D  & E \\
  \hline
1 	& 212 & 414 & 4 		& 23 & fgw	\\
2 	& 212 & 414 & v 		& 23 & fgw	\\
3 	& 212 & 414 & vfwe		& 23 & 长一些的内容	\\
4 	& 212 & 414 & 4fwe		& 23 & 嗯	\\
5 	& af2 & 4vx & 4 		& 23 & 长一些的内容	\\
6 	& af2 & 4vx & 4 		& 23 & fgw	\\
7 	& 212 & 414 & 4 		& 23 & fgw	\\

\hline{}
\end{tabular}
\end{table}

\emph{表格如表~\ref{T.example} 所示,\LaTeX 表格技巧很多,这里不再详细介绍。}

\lipsum

\newpage    % 两个章节之间分页,不想分的话可注释掉

\section{公式插入示例}

\lipsum

\emph{公式插入示例如公式~\eqref{E.example} 所示。}
\begin{equation}
\gamma_x=
\begin{cases}
  0, & \text{if $|x| \leq \delta$} \\
  x, & \text{otherwise}
\end{cases}
\label{E.example}
\end{equation}


\newpage    % 两个章节之间分页,不想分的话可注释掉

\section{参考文献插入示例}

\LaTeX \cite{lamport1994latex}插入参考文献最方便的方式是使用 \env{bibliography}\cite{pritchard1969statistical}。

大多数出版商的论文页面都会有导出 \format{bib} 格式参考文献的链接,把每个文献的 \format{bib} 放入 \bib{thesis-references.bib},然后用 \oper{bibkey} 即可插入参考文献。

\lipsum

\newpage    % 两个章节之间分页,不想分的话可注释掉


\section{总结与展望}

\noindent{纯数字编号}
\begin{enumerate}
 \item XXXXXXXXXX
 \label{item1}
 \item XXXXXXXXXX
 \item XXXXXXXXXX
\end{enumerate}
罗马编号
\begin{enumerate}[label=(\roman*)]
 \item XXXXXXXXXX
 \label{item2}
 \item XXXXXXXXXX
 \item XXXXXXXXXX
\end{enumerate}
括号编号
\begin{enumerate}[label=(\arabic*)]
 \item XXXXXXXXXX
 \label{item3}
 \item XXXXXXXXXX
 \item XXXXXXXXXX
\end{enumerate}
半括号编号
\begin{enumerate}[label=\arabic*)]
 \item XXXXXXXXXX
 \label{item4}
 \item XXXXXXXXXX
 \item XXXXXXXXXX
\end{enumerate}
小字母编号
\begin{enumerate}[label=\alph*)]
 \item XXXXXXXXXX
 \label{item5}
 \item XXXXXXXXXX
 \item XXXXXXXXXX
\end{enumerate}

引用测试,正如~\ref{item1}、\ref{item2}、\ref{item3}、\ref{item4}、\ref{item5} 所示

\subsection{工作展望}
手动编号 %(不推荐,无法被交叉引用)

本课题针对XX,鉴于XXX,对XX进行了提高,但是XXX,所以有如下XX:

(1)目前XX虽然XX,但是XX仍然XX,所以XX仍然是一个值得XX的问题。

(2)随着XX,XX具有XX的问题,仍值得进一步XX。

(3)本课题在XX有了XX,但是XX的XX还存在XX,所以XX。


\newpage
