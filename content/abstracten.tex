%!TEX root = ../sztuthesis_main.tex
\phantomsection
\addcontentsline{toc}{section}{Abstract}

\keywordsen{SZTU;~ECG acquisition;Wireless synchronization;Multiple devices;Time synchronization}
\categoryen{TP391}
\begin{abstracten}
    The rising incidence of cardiovascular diseases underscores the critical need for advancements in ECG monitoring technology. Traditional ECG devices face limitations in portability and real-time performance, particularly in scenarios requiring continuous wearable monitoring. This project focuses on addressing key challenges in multi-device ECG signal acquisition, including device scalability, time synchronization, and concurrent measurement. It achieves low-latency, high-speed synchronized acquisition for multiple ECG devices within a local area network.  

    The system integrates three components: node devices for ECG acquisition and RF transmission, a base station for clock synchronization, network management, and basic data visualization via the LVGL graphical framework, and a QT-based PC software for centralized data monitoring and control. The node devices employ ESP32-C3 chips to collect ECG signals, with an analog front-end circuit filtering and amplifying signals to maintain amplitudes between 1.5-2.5V. The amplified signals are sampled by the ADC at approximately 1000Hz and transmitted wirelessly. The base station ensures precise synchronization across devices, while the PC software enables real-time data display, storage, and system configuration.  
    
    Testing demonstrates a practical sampling rate of 990Hz, clear ECG waveform reconstruction, and an average node power consumption of 100mA. Powered by two 10400mAh batteries, the system sustains 6 hours of continuous operation. This work provides a viable framework for wireless synchronized ECG acquisition, offering significant potential for improving cardiovascular disease diagnosis and long-term health monitoring.
\end{abstracten}