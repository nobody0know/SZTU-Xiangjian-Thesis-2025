%!TEX root = ../sztuthesis_main.tex
\phantomsection
\addcontentsline{toc}{section}{Abstract}

\keywordsen{ECG acquisition;Wireless synchronization;Multiple devices;Time synchronization}
\categoryen{TP391}
\begin{abstracten}
    The rising incidence of cardiovascular diseases highlights the importance of advancements in electrocardiogram (ECG) monitoring technologies. Traditional ECG machines face limitations in convenience and real-time performance, especially in scenarios requiring continuous and portable monitoring.

    This project addresses key challenges in multi-device ECG signal acquisition, particularly in terms of device quantity, time synchronization, and concurrent measurement. It successfully achieves low-latency, high-speed synchronized acquisition across numerous ECG devices within a local area network (LAN).
    
    The system comprises node devices for ECG acquisition and wireless RF transmission, a base station device for clock synchronization, networking, and basic data display using the LVGL graphical framework, and a graphical upper computer software developed with QT for data collection, monitoring, and control. The node devices utilize the ESP32-C3 chip for ECG signal acquisition and transmission. After onboard hardware preprocessing, including analog front-end filtering and amplification, the ECG signal amplitude is adjusted to 1.5-2.5V. The amplified signal is sampled via the ADC module of the ESP32-C3 at a rate of approximately 1000Hz and transmitted. The base station provides clock signals and network management, while the upper computer software offers flexible device configuration, data visualization, and storage.
    
    Through hardware-software co-design and optimization, this research successfully constructs a wireless ECG monitoring system supporting multi-node collaborative acquisition. The system innovatively adopts a dual-layer network architecture combining ESP-NOW-based provisioning and UDP transmission, achieving millisecond-level time synchronization (synchronization error <±2ms) among multiple devices in a LAN environment, thereby overcoming the spatial and temporal limitations of traditional single-device acquisition. Test results demonstrate that the system maintains an effective sampling rate of 990Hz under a 15-node concurrent scenario, with a UDP packet loss rate below 0.01\% and a timestamp continuity rate exceeding 99\%. By introducing adaptive filtering algorithms and right-leg drive technology, the system achieves a signal-to-noise ratio (SNR) of -19.66dB in complex electromagnetic environments, significantly outperforming similar wireless ECG devices (typical value: -15dB). In terms of power management, the innovative integration of Auto Light-Sleep mode and dynamic voltage frequency scaling enables the node devices to achieve 6 hours of continuous monitoring with an average working current of 100mA, extending battery life by 40\% compared to traditional solutions.
    
    This technical solution demonstrates unique advantages in exercise stress testing and sleep monitoring scenarios. The system supports CSV data export, providing high-quality data sources for remote ECG monitoring and AI-assisted diagnosis. The research outcomes not only drive technological innovation in wearable medical devices but also provide critical technical support for constructing regional ECG monitoring networks and implementing group prevention and control of cardiovascular diseases. This contributes positively to the development of a hierarchical medical system and the downward extension of medical resources.
    
\end{abstracten}