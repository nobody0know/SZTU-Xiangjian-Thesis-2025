%!TEX root = ../sztuthesis_main.tex
\phantomsection
\addcontentsline{toc}{section}{Abstract}

\keywordsen{SZTU;~ECG acquisition;Wireless synchronization;Multiple devices;Time synchronization}
\categoryen{TP391}
\begin{abstracten}
    The increasing prevalence of cardiovascular diseases highlights the critical importance of electrocardiogram (ECG) monitoring, while traditional ECG detection technologies exhibit limitations in portability, accuracy, and real-time performance. This study focuses on resolving time synchronization and concurrent measurement challenges in multi-device ECG signal acquisition, successfully implementing a low-latency, high-rate synchronous acquisition system for multiple ECG devices within a local area network.
    The system comprises three components: ECG acquisition/transmission node devices, clock synchronization/networking base station equipment, and host computer software for data collection, monitoring, and control. Node devices employ the ESP32C3 microcontroller to collect and transmit ECG signals after filtering and amplification, achieving a sampling rate of approximately 1000 Hz. The base station manages clock synchronization and network coordination, while the host software enables flexible device configuration, real-time data visualization, and storage.
    Experimental results demonstrate a stable 1000 Hz data acquisition rate with distinct ECG waveform resolution. The node devices operate at an average current of 100 mA, sustaining 6-hour continuous operation using two 10400 batteries. This research presents an effective wireless synchronous ECG acquisition framework, offering significant potential for advancing cardiovascular disease diagnosis technologies.
\end{abstracten}